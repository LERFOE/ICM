% =========================
% 最终版:竞技资产状态空间 \mathcal{R}_t(按“10个核心变量”精简版)
% =========================
\begin{comment}
    

\section{最终版竞技资产状态空间 $\mathcal{R}_t$}

\subsection{设计原则}
\begin{itemize}
    \item \textbf{可求解性优先}:用离散分桶/低维向量刻画竞技状态,避免球员逐项指标导致维数灾难。
    \item \textbf{决策相关性}:仅保留对“建队/交易/续约/轮换/杠杆承受能力”有直接影响的竞技变量。
    \item \textbf{可从公开数据映射}:每个变量给出明确的计算方式,保证可由赛季数据或滚动窗口统计得到。
\end{itemize}

\subsection{竞技状态向量的总体形式}

令球队在时刻 $t$ 的竞技资产状态为一个由 $10$ 个核心变量组成的向量:
\begin{equation}
\mathcal{R}_t
=
\Big\{
\mathbf{Q}_t,\;
\mathbf{C}_t,\;
\mathbf{P}_t,\;
\mathbf{L}_t,\;
\mathbf{A}_t,\;
\mathbf{W}_t,\;
\text{ELO}_t,\;
\text{Syn}_t,\;
\mathbf{O}_t,\;
\text{SOS}_t
\Big\}.
\end{equation}

其中前 $1$--$5$ 项刻画球员资产结构(可控内生),第 $6$--$8$ 项刻画球队竞技表现与体系(半内生),第 $9$--$10$ 项刻画赛程对抗环境(外生)。

% -------------------------
% 1) 球员技术能力:低维综合向量
% -------------------------
\subsubsection{1. 球员技术能力聚合向量 $\mathbf{Q}_t$(Team Skill Aggregate)}

为避免逐球员逐指标带来高维状态,先定义每名球员 $i$ 的\textbf{标准化特征}:
\begin{equation}
z_{i,k}
=
\frac{x_{i,k}-\mu_k}{\sigma_k},
\qquad k \in \mathcal{K},
\end{equation}
其中 $x_{i,k}$ 为球员指标(如 TS\%、AST、TOV、DRTG、BPM 等),$\mu_k,\sigma_k$ 为同赛季联盟均值与标准差。

将球员指标压缩为 $d$ 维\textbf{能力因子}(建议 $d=4\sim 6$):
\begin{equation}
\mathbf{q}_i
=
\mathbf{W}\mathbf{z}_i
\in \mathbb{R}^{d},
\end{equation}
其中 $\mathbf{z}_i=[z_{i,k}]_{k\in\mathcal{K}}$,$\mathbf{W}$ 可由 PCA/因子分析得到,或由专家权重设定。

球队层面的能力聚合向量定义为\textbf{上场时间加权平均}:
\begin{equation}
\mathbf{Q}_t
=
\sum_{i\in \mathcal{R}(t)}
\omega_{i,t}\,\mathbf{q}_i,
\qquad
\omega_{i,t}=\frac{\text{MP}_{i,t}}{\sum_{j\in \mathcal{R}(t)} \text{MP}_{j,t}},
\end{equation}
其中 $\mathcal{R}(t)$ 为当期大名单集合,$\text{MP}_{i,t}$ 为球员在滚动窗口(如最近 $g$ 场)内平均上场时间。

\textbf{解释(示例)}:可取
$\mathbf{q}_i=[q_{i,\text{Off}},q_{i,\text{Def}},q_{i,\text{Play}},q_{i,\text{Reb}}]$
分别代表进攻/防守/组织/篮板四个综合因子。

% -------------------------
% 2) 球员类型分布:聚类标签的直方图
% -------------------------
\subsubsection{2. 球员类型聚类分布 $\mathbf{C}_t$(Roster Archetype Mix)}

对联盟球员做聚类得到类型标签 $c_i\in\{1,\dots,K\}$(建议 $K=6\sim 10$)。定义球队类型构成向量为:
\begin{equation}
\mathbf{C}_t
=
[c_{t,1},\dots,c_{t,K}],
\qquad
c_{t,k}=\sum_{i\in \mathcal{R}(t)} \mathbb{I}(c_i=k).
\end{equation}

该向量刻画“阵容原型结构”(例如:空间型射手、护框中锋、持球核心、3D 侧翼等),用于约束阵容搭配与交易方向。

% -------------------------
% 3) 位置平衡:人数向量 + 约束
% -------------------------
\subsubsection{3. 位置平衡约束向量 $\mathbf{P}_t$(Positional Balance)}

\begin{equation}
\mathbf{P}_t
=
[p_{\text{PG}},p_{\text{SG}},p_{\text{SF}},p_{\text{PF}},p_{\text{C}}],
\qquad
p_{\text{pos}}
=\sum_{i\in \mathcal{R}(t)}\mathbb{I}(\text{pos}_i=\text{pos}).
\end{equation}

\textbf{基本约束}:
\begin{equation}
\sum_{\text{pos}} p_{\text{pos}}=N_{\text{roster}},
\qquad
p_{\text{pos}}\ge 1,\ \forall \text{pos}.
\end{equation}

(若考虑位置模糊,可用“主位置+副位置”将 $\mathbb{I}(\cdot)$ 改为软计数权重。)

% -------------------------
% 4) 合同状态:到期结构的低维化
% -------------------------
\subsubsection{4. 合同状态向量 $\mathbf{L}_t$(Contract Maturity Profile)}

为避免逐球员合同年限带来高维,采用\textbf{到期结构分桶}表示:
\begin{equation}
\mathbf{L}_t
=
[\ell_{0,t},\ell_{1,t},\ell_{2+,t}],
\end{equation}
其中
\begin{equation}
\ell_{0,t}=\sum_{i}\mathbb{I}(l_i=0),\quad
\ell_{1,t}=\sum_{i}\mathbb{I}(l_i=1),\quad
\ell_{2+,t}=\sum_{i}\mathbb{I}(l_i\ge 2),
\end{equation}
$l_i$ 为球员 $i$ 剩余合同年限(年)。

\textbf{可选增强}(若你希望更精确且仍可控):加入“保障/非保障/选项”比例,但仍保持分桶维度常数。

% -------------------------
% 5) 年龄曲线:年龄结构的低维表示
% -------------------------
\subsubsection{5. 年龄结构向量 $\mathbf{A}_t$(Age Curve Profile)}

采用年龄分布的低维表示:
\begin{equation}
\mathbf{A}_t
=
[a_{\text{mean},t},\ a_{\text{var},t},\ a_{28+,t}],
\end{equation}
其中
\begin{equation}
a_{\text{mean},t}=\frac{1}{N_{\text{roster}}}\sum_{i} a_i,\quad
a_{\text{var},t}=\frac{1}{N_{\text{roster}}}\sum_{i}(a_i-a_{\text{mean},t})^2,\quad
a_{28+,t}=\sum_{i}\mathbb{I}(a_i\ge 28).
\end{equation}

该向量用于刻画“即战力 vs 衰退风险”的结构特征,并可在状态转移中引入年龄衰退函数。

% -------------------------
% 6) 战绩状态:滚动窗口战绩(分桶)
% -------------------------
\subsubsection{6. 当前赛季战绩状态 $\mathbf{W}_t$(Performance Record State)}

为使状态平稳且可比较,使用滚动窗口(最近 $g$ 场)定义:
\begin{equation}
\mathbf{W}_t
=
[\text{Win\%}_t^{(g)},\ \text{Streak}_t,\ \text{RankBin}_t],
\end{equation}
其中 $\text{Win\%}_t^{(g)}$ 为最近 $g$ 场胜率,$\text{Streak}_t$ 为连胜/连败长度,
$\text{RankBin}_t$ 为排名分桶(如:前 1/3、中 1/3、后 1/3)。

% -------------------------
% 7) 战力评级:ELO(或 SRS)单指标
% -------------------------
\subsubsection{7. 球队战力评级 $\text{ELO}_t$(Team Strength Rating)}

采用 ELO 作为单一强度指标:
\begin{equation}
\text{ELO}_{t+1}
=
\text{ELO}_t
+
K_{\text{elo}}\left(W_{\text{actual},t}-W_{\text{expected},t}\right),
\end{equation}
其中 $W_{\text{expected},t}$ 可由 ELO 差异的 logistic 期望胜率给出。

% -------------------------
% 8) 阵容协同:量化“1+1>2”
% -------------------------
\subsubsection{8. 阵容协同效应 $\text{Syn}_t$(Roster Synergy)}

用“实际阵容净效率”减去“按球员加总的基准净效率”定义协同增益:
\begin{equation}
\text{Syn}_t
=
\underbrace{\text{NetRTG}_t^{(g)}}_{\text{阵容整体表现}}
-
\underbrace{\sum_{i\in \mathcal{R}(t)}\omega_{i,t}\,\text{BPM}_{i,t}}_{\text{个体贡献线性叠加基准}},
\end{equation}
其中
$\text{NetRTG}_t^{(g)}=\text{ORTG}_t^{(g)}-\text{DRTG}_t^{(g)}$
为最近 $g$ 场的净效率,$\text{BPM}_{i,t}$ 为球员 box plus-minus(同样可用滚动窗口),
$\omega_{i,t}$ 为上场时间权重。

\textbf{解释}:$\text{Syn}_t>0$ 表示体系加成(“1+1>2”),$\text{Syn}_t<0$ 表示风格冲突或轮换失配。

% -------------------------
% 9) 对手强度分布:剩余赛程 ELO 列表(可再压缩)
% -------------------------
\subsubsection{9. 对手强度分布 $\tilde{\mathbf{O}}_t$(Opponent Strength Statistics)}

原始赛程长度 $m$ 为变量,为保证状态空间维数固定,定义统计量向量:
\begin{equation}
\tilde{\mathbf{O}}_t = [\mu_{\text{opp}}, \sigma_{\text{opp}}, \text{N}_{\text{elite}}]
\end{equation}
其中 $\mu_{\text{opp}}$ 为剩余对手平均 ELO,$\sigma_{\text{opp}}$ 为标准差,$\text{N}_{\text{elite}}$ 为剩余赛程中强队(ELO $>1600$)的数量。

\textbf{注}:为保持状态维度常数,可用统计量替代原始向量:
\begin{equation}
\tilde{\mathbf{O}}_t=
[\overline{o}_t,\ \text{Var}(o)_t,\ o^{(90\%)}_t],
\end{equation}
分别为均值、方差与 90\% 分位数(“强敌尾部风险”)。

% -------------------------
% 10) 赛程难度:SOS(单指标)
% -------------------------
\subsubsection{10. 赛程难度系数 $\text{SOS}_t$(Strength of Schedule)}

\begin{equation}
\text{SOS}_t
=
\frac{1}{m}\sum_{j=1}^{m}\frac{o_j}{\text{ELO}_{\text{avg}}},
\end{equation}
其中 $\text{ELO}_{\text{avg}}$ 为联盟平均 ELO。

(可选)若需体现压缩赛程惩罚,可在环境状态中另加背靠背数 $N_{\text{B2B}}$,或将其并入 $\text{SOS}_t$ 的修正项。

% -------------------------
% 数学表示总结(竞技状态仅此 10 项)
% -------------------------
\subsection{竞技状态空间总结}

最终,竞技资产状态空间采用常数维、可计算、可用于动态规划/ADP/MCTS 的表示:
\begin{equation}
\boxed{
\mathcal{R}_t
=
\Big\{
\mathbf{Q}_t,\;
\mathbf{C}_t,\;
\mathbf{P}_t,\;
\mathbf{L}_t,\;
\mathbf{A}_t,\;
\mathbf{W}_t,\;
\text{ELO}_t,\;
\text{Syn}_t,\;
\tilde{\mathbf{O}}_t,\;
\text{SOS}_t
\Big\}.
}
\end{equation}

\textbf{维度说明(示例)}:若 $\mathbf{Q}_t\in\mathbb{R}^4$,$\mathbf{C}_t\in\mathbb{R}^K$($K=8$),则总体维度约为
$4+8+5+3+3+3+1+1+3+1=32$,
其中 $\tilde{\mathbf{O}}_t$ 为3维统计量,保持状态空间维数恒定,显著提升可求解性。
\section{建模建议}

\begin{enumerate}
    \item \textbf{降维处理}:使用主成分分析(PCA)将技术指标压缩至5-10个综合因子
    \item \textbf{分层建模}:战术层(单场比赛模拟)$\rightarrow$ 赛季层(82场积分)$\rightarrow$ 多年期(资产演化)
    \item \textbf{蒙特卡洛模拟}:对伤病/选秀/交易等随机事件进行1000+次情景推演
    \item \textbf{敏感性分析}:识别核心杠杆变量(如顶薪球员健康度、选秀权价值)
\end{enumerate}

\section{最终版财务状态空间 $\mathcal{F}_t$(精简核心版)}

\subsection{设计原则}
\begin{itemize}
    \item \textbf{只保留直接影响决策且无法从其他变量推导的核心变量}
    \item \textbf{明确区分存量(Balance Sheet)与流量(Cash Flow)}
    \item \textbf{与竞技状态 $\mathcal{R}_t$ 建立清晰的因果映射}
\end{itemize}

\subsection{核心财务状态向量}

$$\mathcal{F}_t = \{\lambda_t, \text{CF}_t, \mathbf{\Psi}_t, \text{Cap}_t, V_t, \text{Equity}_t\}$$

\subsubsection{1. 杠杆率 $\lambda_t \in [0, 1]$}

\textbf{定义}:
\begin{equation}
\lambda_t = \frac{D_t}{D_t + E_t}
\end{equation}

其中:
\begin{itemize}
    \item $D_t$: 总债务存量(银行贷款、债券等)
    \item $E_t$: 所有者权益(球队市场估值)
\end{itemize}

\textbf{状态分级}(用于策略分支):
\begin{itemize}
    \item Safe: $\lambda < 0.3$(可积极投资,融资成本低)
    \item Balanced: $0.3 \leq \lambda < 0.6$(需谨慎管理)
    \item Distressed: $\lambda \geq 0.6$(被迫收缩,面临强制出售风险)
\end{itemize}

\textbf{转移机制}:
\begin{equation}
\lambda_{t+1} = \frac{D_t + \Delta D_t - \text{Repayment}_t}{(D_t + \Delta D_t - \text{Repayment}_t) + (E_t + \Delta E_t)}
\end{equation}

其中:
\begin{itemize}
    \item $\Delta D_t$: 当期新增债务(动作变量)
    \item $\text{Repayment}_t$: 债务偿还
    \item $\Delta E_t$: 估值变化(受竞技成绩和媒体周期影响)
\end{itemize}

\textbf{利息成本}:
\begin{equation}
\text{Interest}_t = D_t \times r_{\text{debt},t}
\end{equation}

其中债务利率随杠杆率上升:
\begin{equation}
r_{\text{debt},t} = r_{\text{risk-free}} + \beta \cdot \lambda_t + \delta \cdot \mathbb{I}(\text{CF}_t < 0)
\end{equation}

$\beta \approx 0.05$(杠杆惩罚系数),$\delta \approx 0.02$(流动性危机额外溢价)

\textbf{备注}:这是题目要求的"leverage调整"的核心量化指标。

---

\subsubsection{2. 经营性现金流 $\text{CF}_t \in \mathbb{R}$}

\textbf{定义}(简化的现金流瀑布):
\begin{equation}
\text{CF}_t = \text{Rev}_t - \text{Cost}_t - \text{Tax}_t - \text{Interest}_t
\end{equation}

\textbf{收入总额}(二元结构):
\begin{equation}
\text{Rev}_t = \underbrace{\text{Rev}_{\text{Performance}}}_{\text{竞技驱动}} + \underbrace{\text{Rev}_{\text{Star}}}_{\text{人气驱动}}
\end{equation}

\begin{itemize}
    \item $\text{Rev}_{\text{Performance}} = \alpha_1 \cdot \text{Win\%}_t + \alpha_2 \cdot \text{Playoff}_t$
    
    其中 $\alpha_1 \approx 2$M(胜率每提升10\%带来200万美元收入),$\alpha_2 \in [0.5, 5]$M(季后赛轮次奖金)
    
    \item $\text{Rev}_{\text{Star}} = \beta_1 \cdot \text{StarPower}_t + \beta_2 \cdot \text{MediaExposure}_t$
    (注:$\text{StarPower}$ 与 $\text{MediaExposure}$ 为基于 $\mathcal{R}_t$ 与 $\mathcal{E}_t$ 的衍生观测变量,不作为独立状态维度)
    
    其中 $\beta_1 \approx 1.5$M(顶级球星带来150万美元额外收入),$\beta_2$(全国转播场次系数)
\end{itemize}

\textbf{成本总额}(三大块):
\begin{equation}
\text{Cost}_t = \text{Salary}_{\text{Total}} + \text{Ops}_{\text{Fixed}} + \text{Venue}_t
\end{equation}

\begin{itemize}
    \item $\text{Salary}_{\text{Total}} = \sum_{i=1}^{N_{\text{roster}}} \text{Salary}_i$(所有球员合同总和,直接来自财务状态 $\mathbf{\Psi}_t$)

    \item $\text{Ops}_{\text{Fixed}}$: 固定运营成本(教练、管理层、差旅,约3-5M美元/年)
    \item $\text{Venue}_t$: 场馆成本
    \begin{equation}
    \text{Venue}_t = \begin{cases}
    \text{Rent}_{\text{annual}}, & \text{if 租赁模式} \\
    \frac{\text{VenueValue}}{T_{\text{depreciation}}} + \text{Maintenance}_t, & \text{if 自有模式}
    \end{cases}
    \end{equation}
\end{itemize}

\textbf{税收支出}(简化):
\begin{equation}
\text{Tax}_t = \max\left(0, (\text{Rev}_t - \text{Cost}_t - \text{Interest}_t) \times r_{\text{corporate}}\right)
\end{equation}

其中 $r_{\text{corporate}} \approx 25\%$(联邦+州综合税率)

\textbf{关键阈值}:
\begin{itemize}
    \item $\text{CF}_t > 0$: 健康状态,可积累现金储备
    \item $\text{CF}_t < 0$: 亏损状态,需动用储备或融资
    \item $\text{CF}_t < -C_{\text{min}}$: 触发"强制出售"吸收态(如连续3年亏损超5M)
\end{itemize}

\textbf{现金储备演化}:
\begin{equation}
\text{Cash}_{t+1} = \text{Cash}_t + \text{CF}_t + \Delta D_t - \text{Repayment}_t - \text{Dividends}_t
\end{equation}

\textbf{备注}:这是短期生存能力的核心指标,决定球队能否"熬过"重建期。

---

\subsubsection{3. 薪资与合同结构向量 $\mathbf{\Psi}_t$(Salary Structure)}

\begin{equation}
\mathbf{\Psi}_t = [\mu_{S,t},\ \sigma_{S,t},\ r_{\text{max},t},\ S_{\text{commit},t}]
\end{equation}

\begin{itemize}
    \item $\mu_{S,t}$ (\textbf{Salary Level}):平均薪资水平,作为计算总薪资的基准。
    \item $\sigma_{S,t}$ (\textbf{Salary Distribution}):薪资分布特征(如方差或基尼系数),反映薪资结构的集中度(避免“中产陷阱”)。
    \item $r_{\text{max},t}$ (\textbf{Max Salary Ratio}):顶薪球员薪资占比,衡量头部球星对空间的占用。
    \item $S_{\text{commit},t}$ (\textbf{Contract Amount}):未来保障合同总金额,反映长期财务刚性。
\end{itemize}

---

\subsubsection{4. 工资帽状态 $\text{Cap}_t = \{S_{\text{used}}, S_{\text{avail}}, \text{Tax}_{\text{Status}}\}$}

\textbf{已用薪资总额}:
\begin{equation}
S_{\text{used}} = \sum_{i=1}^{N_{\text{roster}}} \text{Salary}_i
\end{equation}

\textbf{可用薪资空间}:
\begin{equation}
S_{\text{avail}} = \text{Cap}_{\text{limit}} - S_{\text{used}}
\end{equation}

\textbf{税收状态}(离散化为3档):
\begin{equation}
\text{Tax}_{\text{Status}} = \begin{cases}
0 \quad (\text{Under Cap}), & \text{if } S_{\text{used}} < \text{Cap}_{\text{limit}} \\
1 \quad (\text{Taxpayer}), & \text{if } \text{Cap}_{\text{limit}} \leq S_{\text{used}} < \text{Apron}_1 \\
2 \quad (\text{Hard Apron}), & \text{if } S_{\text{used}} \geq \text{Apron}_2
\end{cases}
\end{equation}

\textbf{奢侈税惩罚}(若适用):
\begin{equation}
\text{LuxuryTax}_t = \begin{cases}
0, & \text{if Tax\_Status} = 0 \\
\sum_{k} \text{Bracket}_k \times \text{Rate}_k, & \text{if Tax\_Status} \geq 1
\end{cases}
\end{equation}

其中税率为累进制:
\begin{itemize}
    \item 首个\$5M超出部分:税率1.5倍(\$1.50/\$1.00)
    \item 次个\$5M:税率2.0倍
    \item 超过\$10M:税率3.0倍以上
    \item 若连续3年纳税(Repeater Tax):税率再乘1.5倍
\end{itemize}

\textbf{动作空间约束}:
\begin{itemize}
    \item Tax\_Status = 0: 可全额使用中产特例,可大额交易
    \item Tax\_Status = 1: 部分交易特例失效
    \item Tax\_Status = 2: 禁止打包交易、首轮签交易受限、无中产特例
\end{itemize}

\textbf{工资帽增长}(来自环境状态):
\begin{equation}
\text{Cap}_{\text{limit},t+1} = \text{Cap}_{\text{limit},t} \times (1 + g_{\text{cap}})
\end{equation}

其中 $g_{\text{cap}} \in \{8\%, 12\%, 40\%\}$(平滑增长 vs 媒体合约跳跃)

\textbf{备注}:这是硬约束,直接限制动作空间 $\mathcal{A}_t$。

---

\subsubsection{5. 品牌估值增速 $V_t \in \mathbb{R}$(替代绝对估值)}

\textbf{定义}(相对于上一期的估值变化率,以小数表示):
\begin{equation}
V_t = \frac{E_t - E_{t-1}}{E_{t-1}}
\end{equation}

\textbf{驱动因素}(回归模型):
\begin{equation}
V_t = \gamma_1 \cdot \Delta\text{Win\%} + \gamma_2 \cdot \text{MediaDeal}_{\text{Spike}} + \gamma_3 \cdot \text{StarAcquisition} + \gamma_4 \cdot \text{CF}_t + \epsilon
\end{equation}

参数估计(示例):
\begin{itemize}
    \item $\gamma_1 \approx 0.05$(胜率提升10\%导致估值增长5\%)
    \item $\gamma_2 \approx 0.30$(新媒体合约生效导致估值跳跃30\%)
    \item $\gamma_3 \approx 0.15$(获得超级球星导致估值增长15\%)
    \item $\gamma_4 \approx 0.0002$(每百万美元利润增加0.02\%估值)
\end{itemize}

\textbf{状态分级}:
\begin{itemize}
    \item 高增长($V > 0.20$):估值飙升期(如获得Caitlin Clark)
    \item 稳定($0 < V < 0.20$):正常发展
    \item 贬值($V < 0$):球队价值下滑
\end{itemize}

\textbf{长期估值累积}:
\begin{equation}
E_t = E_0 \times \prod_{\tau=1}^{t} (1 + V_{\tau})
\end{equation}

\textbf{备注}:这捕捉题目要求的"特许经营权品牌资产",用增速而非绝对值避免数值过大。

---

\subsubsection{6. 股权结构与球员权益 $\text{Equity}_t = \{\text{Owner}_{\text{share}}, \text{Player}_{\text{equity}}, \text{Dilution}_t\}$}

\textbf{核心问题}:题目明确提到"球员股权可作为薪资替代方案"(profit participation, equity stake等)

\textbf{所有者权益占比}:
\begin{equation}
\text{Owner}_{\text{share},t} = \frac{E_{\text{owner}}}{E_t} \times 100\%
\end{equation}

初始状态:$\text{Owner}_{\text{share},0} = 100\%$

\textbf{球员股权池}(员工持股计划,ESOP):
\begin{equation}
\text{Player}_{\text{equity},t} = \sum_{i \in \text{Granted}} \omega_i
\end{equation}

其中 $\omega_i$ 为球员 $i$ 的持股比例(通常 $0.1\% \sim 5\%$)

\textbf{股权授予条件}:
\begin{itemize}
    \item \textbf{类型1:收入分成}(Revenue Sharing)
    
    球员 $i$ 获得当年收入的额外分成:
    \begin{equation}
    \text{Bonus}_i = \text{Rev}_t \times \omega_i \times \mathbb{I}(\text{Rev}_t > \text{Threshold})
    \end{equation}
    
    \item \textbf{类型2:利润分成}(Profit Participation)
    
    球员 $i$ 获得当年净利润的分成:
    \begin{equation}
    \text{Bonus}_i = \max(0, \text{CF}_t) \times \omega_i
    \end{equation}
    
    \item \textbf{类型3:长期股权}(Equity Stake)
    
    球员 $i$ 获得球队 $\omega_i\%$ 的永久股权,退役后仍享有:
    \begin{itemize}
        \item 年度分红权(如球队盈利时按比例分配)
        \item 增值收益权(球队出售时按估值分成)
    \end{itemize}
    
    \textbf{估值退出收益}:
    \begin{equation}
    \text{Payout}_i = (E_T - E_0) \times \omega_i \quad \text{(球队出售时)}
    \end{equation}
\end{itemize}

\textbf{股权稀释动态}:
\begin{equation}
\text{Owner}_{\text{share},t+1} = \text{Owner}_{\text{share},t} \times \frac{E_t}{E_t + \Delta E_{\text{new}}}
\end{equation}

当授予新股权或引入外部投资者时发生稀释

\textbf{决策权衡}(动作空间中的选择):

对于顶薪球员 $i$,所有者可选择:
\begin{enumerate}
    \item \textbf{纯现金合同}:
    \begin{equation}
    \text{Salary}_i = \$500k/\text{year}, \quad \omega_i = 0\%
    \end{equation}
    
    \item \textbf{现金+股权混合}:
    \begin{equation}
    \text{Salary}_i = \$350k/\text{year}, \quad \omega_i = 2\%, \quad \text{期望总价值} \approx \$500k
    \end{equation}
    
    优势:当期现金流压力减小(CF\_t改善),但未来利润需分享
    
    \item \textbf{股权激励型}(适用于重建期):
    \begin{equation}
    \text{Salary}_i = \$250k/\text{year}, \quad \omega_i = 5\%, \quad \text{Vesting Period} = 4\text{年}
    \end{equation}
    
    若球队4年内估值从\$100M增至\$200M,球员额外获得:
    \begin{equation}
    \text{Gain}_i = (\$200M - \$100M) \times 5\% = \$5M
    \end{equation}
\end{enumerate}

\textbf{约束条件}:
\begin{itemize}
    \item \textbf{总股权池上限}:
    \begin{equation}
    \sum_i \omega_i \leq \omega_{\max} \quad (\omega_{\max} \approx 10\%-15\%)
    \end{equation}
    
    避免过度稀释导致所有者失去控制权
    
    \item \textbf{流动性要求}:
    
    股权计划不能导致现金流枯竭:
    \begin{equation}
    \text{Cash}_t - \text{Salary}_{\text{Total}} \geq C_{\min}
    \end{equation}
    
    \item \textbf{球员接受度}:
    
    只有当期望总价值≥纯现金合同时,球员才接受股权方案:
    \begin{equation}
    \text{Salary}_i + \mathbb{E}[\text{Equity Gain}_i] \geq \text{MarketValue}_i
    \end{equation}
\end{itemize}

\textbf{股权退出机制}(Liquidity Events)}:

\begin{enumerate}
    \item \textbf{球队出售}:
    
    所有股权持有者按比例分享交易对价
    
    \item \textbf{回购条款}(Buyback):
    
    球员退役/交易时,球队有权按公允价值回购股权:
    \begin{equation}
    \text{Buyback Price} = E_t \times \omega_i \times (1 - \text{Discount})
    \end{equation}
    
    Discount通常10-20\%(流动性折扣)
    
    \item \textbf{分红政策}:
    
    若球队选择不分红(将利润全部再投资),股权持有者无当期收益,只能等待长期增值
\end{enumerate}

\textbf{财务影响建模}:

\textbf{对现金流的影响}:
\begin{equation}
\text{CF}_t^{\text{with equity}} = \text{CF}_t^{\text{no equity}} + \underbrace{(\text{Salary}_{\text{saved}} - \text{Dividends}_{\text{paid}})}_{\text{净效应}}
\end{equation}

\begin{itemize}
    \item 短期:节省现金工资,CF\_t改善
    \item 长期:需支付分红或面临回购压力,CF\_t恶化
\end{itemize}

\textbf{对估值的影响}:
\begin{equation}
E_t^{\text{diluted}} = E_t \times (1 - \sum_i \omega_i)
\end{equation}

所有者实际拥有的价值被稀释

\textbf{最优化问题}(所有者视角):

在给定薪资帽约束下,选择股权授予比例以最大化期望净现值:
\begin{equation}
\max_{\{\omega_i\}} \mathbb{E}\left[ \sum_{t=0}^{T} \frac{\text{CF}_t \times (1 - \sum_i \omega_i)}{(1+r)^t} + \frac{E_T \times (1 - \sum_i \omega_i)}{(1+r)^T} \right]
\end{equation}

约束条件:
\begin{align}
&\sum_i \text{Salary}_i \leq \text{Cap}_{\text{limit}} \\
&\sum_i \omega_i \leq \omega_{\max} \\
&\text{Salary}_i + \mathbb{E}[\text{Equity Gain}_i] \geq \text{MarketValue}_i \quad \forall i
\end{align}

\textbf{备注}:股权激励在初创/重建期球队中特别有效,因为估值增长空间大,球员愿意"赌未来"。

---

\subsection{财务状态的极简表示}

\begin{equation}
\mathcal{F}_t = \begin{cases}
\lambda_t \in [0, 1] & \text{(杠杆率)} \\
\text{CF}_t \in \mathbb{R} & \text{(经营现金流)} \\
\mathbf{\Psi}_t \in \mathbb{R}^4 & \text{(薪资结构:均值/方差/顶薪比/保障金额)} \\
S_{\text{avail}} \in [0, \text{Cap}_{\text{limit}}] & \text{(可用薪资空间)} \\
\text{Tax}_{\text{Status}} \in \{0, 1, 2\} & \text{(税收档位)} \\
V_t \in \mathbb{R} & \text{(品牌估值增速)} \\
\text{Equity}_t \in \mathbb{R}^2 & \text{(所有者权益占比与球员持股池)}
\end{cases}
\end{equation}

\textbf{总维度}:约9维(与 Section 10 统一状态空间保持一致)。

---

\section{最终版环境状态空间 $\mathcal{E}_t$(精简核心版)}

\subsection{设计原则}
\begin{itemize}
    \item \textbf{只保留"外生冲击"和"不可控约束"}
    \item \textbf{宏观经济用单一指标代理}
    \item \textbf{联盟政策聚焦于"阶跃式变化"(如扩军、工资帽暴涨)}
\end{itemize}

\subsection{核心环境状态向量}

$$\mathcal{E}_t = \{\text{Macro}_t, \text{League}_t, \text{Market}_t, \text{FA}_{\text{Market},t}\}$$

---

\subsubsection{1. 宏观经济景气度 $\text{Macro}_t \in \{\text{衰退}, \text{正常}, \text{繁荣}\}$}

\textbf{定义}(离散化的经济周期指标):
\begin{itemize}
    \item \textbf{衰退}:消费者支出下降,门票收入-20\%,赞助合同难续约
    \item \textbf{正常}:基准状态
    \item \textbf{繁荣}:娱乐消费旺盛,门票溢价+15\%,新赞助商涌入
\end{itemize}

\textbf{转移概率}(马尔可夫链):
\begin{equation}
P(\text{Macro}_{t+1} | \text{Macro}_t) = \begin{bmatrix}
0.7 & 0.25 & 0.05 \\
0.15 & 0.7 & 0.15 \\
0.05 & 0.3 & 0.65
\end{bmatrix}
\end{equation}

行顺序:\{衰退, 正常, 繁荣\}

\textbf{影响机制}:
\begin{equation}
\text{Rev}_t = \text{Rev}_{\text{base}} \times \begin{cases}
0.8, & \text{if Macro}_t = \text{衰退} \\
1.0, & \text{if Macro}_t = \text{正常} \\
1.15, & \text{if Macro}_t = \text{繁荣}
\end{cases}
\end{equation}

\textbf{备注}:用三状态马尔可夫链替代复杂的GDP/CPI/利率多维建模。

---

\subsubsection{2. 联盟政策周期 $\text{League}_t = \{\text{Cap}_{\text{Growth}}, I_{\text{Expansion}}, T_{\text{MediaDeal}}\}$}

\textbf{工资帽增长模式 Cap\_Growth}:
\begin{equation}
\text{Cap}_{\text{Growth}} \in \{\text{平滑}, \text{跳跃}\}
\end{equation}

\begin{itemize}
    \item \textbf{平滑增长}(Smooth):每年 $g_{\text{cap}} = 8\%-12\%$(可预测)
    \item \textbf{跳跃增长}(Spike):某年突增 $g_{\text{cap}} = 40\%$(2026年WNBA新媒体合约)
\end{itemize}

\textbf{工资帽演化}:
\begin{equation}
\text{Cap}_{\text{limit},t+1} = \text{Cap}_{\text{limit},t} \times (1 + g_{\text{cap}})
\end{equation}

---

\textbf{扩军时间表 $I_{\text{Expansion}}$}:
\begin{equation}
I_{\text{Expansion},t} = \begin{cases}
1, & \text{if } t \in \{2026, 2028, ...\} \\
0, & \text{otherwise}
\end{cases}
\end{equation}

\textbf{扩军影响}(一次性冲击):

\begin{enumerate}
    \item \textbf{扩军费分红}:
    \begin{equation}
    \text{CF}_t = \text{CF}_t + \frac{\text{ExpansionFee}}{N_{\text{teams}}} \quad \text{(一次性+5M美元)}
    \end{equation}
    
    \item \textbf{人才稀释}:
    
    自由市场球员平均技能下降:
    \begin{equation}
    \mathbb{E}[S_i^{\text{FA}}] = \mathbb{E}[S_i^{\text{FA}}]_{\text{baseline}} \times 0.95
    \end{equation}
    
    \item \textbf{保护名单要求}:
    
    每队需提交保护名单 $P_{\text{Protected}} \subseteq \{1, 2, ..., 12\}$,通常规则:
    \begin{equation}
    |P_{\text{Protected}}| = 6 \sim 8
    \end{equation}
    
    未保护球员可被新球队选走(每队最多失去1名球员)
\end{enumerate}

---

\textbf{媒体合约倒计时 $T_{\text{MediaDeal}}$}:
\begin{equation}
T_{\text{MediaDeal}} \in \{5, 4, 3, 2, 1, 0\}
\end{equation}

\textbf{合约生效触发}:

当 $T_{\text{MediaDeal}} = 0$ 时触发"估值跳跃":
\begin{equation}
V_t = V_{t-1} + \text{Spike}_{\text{media}} \quad (\text{Spike} \approx 30\%-50\%)
\end{equation}

同时:
\begin{equation}
\text{Rev}_{\text{media},t} = \text{Rev}_{\text{media},t-1} \times (1 + g_{\text{media}})
\end{equation}

其中 $g_{\text{media}} \approx 200\%-300\%$(2026年WNBA新合约预期)

\textbf{倒计时更新}:
\begin{equation}
T_{\text{MediaDeal},t+1} = \begin{cases}
T_{\text{MediaDeal},t} - 1, & \text{if } T_{\text{MediaDeal},t} > 0 \\
10, & \text{if } T_{\text{MediaDeal},t} = 0 \quad \text{(重置为下个周期)}
\end{cases}
\end{equation}

\textbf{备注}:这三个子变量共同描述"联盟层面的制度环境"。

---

\subsubsection{3. 本队市场位置 $\text{Market}_t$(准静态参数)}

\textbf{市场规模系数 $\mu_{\text{size}} \in [0.5, 2.0]$}:

\begin{itemize}
    \item \textbf{超大市场}(纽约、洛杉矶):$\mu_{\text{size}} = 2.0$
    \item \textbf{大市场}(芝加哥、旧金山):$\mu_{\text{size}} = 1.5$
    \item \textbf{中等市场}(西雅图、凤凰城):$\mu_{\text{size}} = 1.0$
    \item \textbf{小市场}(印第安纳、康涅狄格):$\mu_{\text{size}} = 0.6$
\end{itemize}

\textbf{影响公式}:
\begin{equation}
\text{Rev}_{\text{gate}} = \text{Base} \times \mu_{\text{size}} \times f(\text{Win\%}, \text{StarPower})
\end{equation}

---

\textbf{本地竞争强度 $\text{Compete}_{\text{local}} \in \{\text{低}, \text{中}, \text{高}\}$}:

\begin{itemize}
    \item \textbf{低}:本城市无其他主流职业球队(如印第安纳,WNBA是夏季唯一主角)
    \item \textbf{中}:有1-2支竞争球队
    \item \textbf{高}:体育市场高度饱和(纽约9支职业球队)
\end{itemize}

\textbf{修正因子}:
\begin{equation}
\text{MediaShare} = \frac{1}{1 + 0.2 \times N_{\text{competitors}}}
\end{equation}

\textbf{综合市场效应}:
\begin{equation}
\text{Rev}_t = \text{Rev}_{\text{base}} \times \mu_{\text{size}} \times \text{MediaShare}
\end{equation}

\textbf{备注}:$\mu_{\text{size}}$ 和 $\text{Compete}_{\text{local}}$ 在球队建立时确定,不随时间变化(因此严格说是"参数"而非"状态",但为了建模完整性放在这里)。

---

\subsubsection{4. 自由市场热度 $\text{FA}_{\text{Market},t} = \{N_{\text{Star}}, \text{Bidding}_t\}$}

\textbf{顶级球星供给数量}:
\begin{equation}
N_{\text{Star}} = |\{i : C_i \in \{\text{核心得分手}, \text{全能侧翼}\} \land l_i = 0\}|
\end{equation}

其中 $C_i$ 为球员类型(来自竞技状态 $\mathcal{R}_t$),$l_i$ 为剩余合同年限

---

\textbf{竞价激烈度}(其他球队的薪资空间总和):
\begin{equation}
\text{Bidding}_t = \sum_{j \neq \text{own}} S_{\text{avail},j}
\end{equation}

\textbf{影响}:
\begin{itemize}
    \item $N_{\text{Star}}$ 高且 $\text{Bidding}$ 低:买方市场,可低价签约
    \item $N_{\text{Star}}$ 低且 $\text{Bidding}$ 高:卖方市场,溢价严重
\end{itemize}

\textbf{签约成本函数}:
\begin{equation}
\text{Contract}_{\text{offer}} = \text{FairValue}_i \times \left(1 + 0.5 \times \frac{\text{Bidding}_t}{\text{Bidding}_{\text{avg}}}\right) \times \left(1 - 0.3 \times \frac{N_{\text{Star}}}{N_{\text{Star,avg}}}\right)
\end{equation}

\textbf{状态转移}:

每个休赛期,自由市场状态重置:
\begin{align}
N_{\text{Star},t+1} &= |\{i : l_i = 0 \text{ at end of season } t\}| \\
\text{Bidding}_{t+1} &= \sum_{j} (\text{Cap}_{\text{limit}} - S_{\text{used},j})
\end{align}

\textbf{备注}:这捕捉"其他球队决策的集体效应",是博弈论元素的简化表达。

---

\subsection{环境状态的极简表示(最终方案)}

\begin{equation}
\mathcal{E}_t = \begin{cases}
\text{Macro}_t \in \{\text{衰退}, \text{正常}, \text{繁荣}\} & \text{(经济周期)} \\
\text{Cap}_{\text{Growth}} \in \{\text{平滑}, \text{跳跃}\} & \text{(工资帽增长模式)} \\
I_{\text{Expansion}} \in \{0, 1\} & \text{(扩军年标记)} \\
T_{\text{MediaDeal}} \in \{0, 1, 2, ..., 5\} & \text{(媒体合约倒计时)} \\
\mu_{\text{size}} \in [0.5, 2.0] & \text{(市场规模系数)} \\
\text{Compete}_{\text{local}} \in \{\text{低}, \text{中}, \text{高}\} & \text{(本地竞争强度)} \\
N_{\text{Star,FA}} \in \mathbb{Z}_+ & \text{(自由市场球星数)} \\
\text{Bidding}_t \in \mathbb{R}_+ & \text{(竞价强度,百万美元)}
\end{cases}
\end{equation}

\textbf{总维度}:8维(相比原方案的15+维大幅精简)

---

\section{状态空间集成设计}

\subsection{完整状态向量 $\mathcal{S}_t = \{\mathcal{R}_t, \mathcal{F}_t, \mathcal{E}_t, \Theta_t\}$}

\begin{itemize}
    \item \textbf{竞技状态 $\mathcal{R}_t$}:10维(球员技能、阵容结构、健康度、选秀权)
    \item \textbf{财务状态 $\mathcal{F}_t$}:9维(杠杆率、现金流、薪资结构、配置状态等)
    \item \textbf{环境状态 $\mathcal{E}_t$}:8维(经济、联盟政策、市场、自由市场)
    \item \textbf{阶段状态 $\Theta_t$}:1维(休赛期/常规赛/交易期/季后赛)
\end{itemize}

\textbf{总维度}:约30维(满足精确控制需求的同时保持计算可行性)

---


\section{MDP 统一状态空间的形式化定义}

为解决跨周期的动态决策问题(涵盖赛季内的滚动决策与休赛期的资产重构),并精确响应联盟赛程规则与扩军事件,系统的总体状态空间定义为竞技、财务、环境三个子空间与阶段状态的笛卡尔积:
\begin{equation}
\mathcal{S} = \mathcal{R} \times \mathcal{F} \times \mathcal{E} \times \Theta
\end{equation}

\subsection{1. 赛季阶段状态 $\Theta_t$}
该变量充当MDP的“决策模式控制器”,决定了模型在当前时刻的可行动作空间 $\mathcal{A}(s_t)$ 及状态转移逻辑:
\begin{equation}
\Theta_t \in \{\text{Offseason},\ \text{Regular},\ \text{TradeDeadline},\ \text{Playoff}\}
\end{equation}
\begin{itemize}
    \item \textbf{Offseason (休赛期)}:决策核心为资产重构。允许选秀(含扩军选秀响应)、自由市场竞价、股权结构调整及长期杠杆规划。
    \item \textbf{Regular (常规赛)}:决策核心为运营优化。允许双向合同签约、票价动态调整、短期伤病特例申请,不可进行核心阵容变更(除非交易期)。
    \item \textbf{TradeDeadline (交易截止日)}:决策核心为博弈。最后调整阵容窗口,面临信息不对称与高溢价风险。
    \item \textbf{Playoff (季后赛)}:决策核心为风险控制。花名册冻结,仅允许战术层面调整与场馆运营优化。
\end{itemize}

\subsection{2. 竞技资产状态 $\mathcal{R}_t$}
采用前文确定的 10 维核心竞技特征向量:
\begin{equation}
\mathcal{R}_t = \Big\{ \mathbf{Q}_t,\; \mathbf{C}_t,\; \mathbf{P}_t,\; \mathbf{L}_t,\; \mathbf{A}_t,\; \mathbf{W}_t,\; \text{ELO}_t,\; \text{Syn}_t,\; \tilde{\mathbf{O}}_t,\; \text{SOS}_t \Big\}
\end{equation}

\subsection{3. 财务状态 $\mathcal{F}_t$}
为了解决 MDP 闭合性问题并统一杠杆定义,我们显式引入\textbf{配置状态} $\mathbf{K}_t$ 和\textbf{所有者权益} $s_t$:
\begin{equation}
\mathcal{F}_t = \Big\{ FV_t,\ D_t,\ \lambda_t,\ \text{CF}_t,\ \mathbf{\Psi}_t,\ S_{\text{avail},t},\ \text{Tax}_{\text{Status},t},\ V_t,\ s_t,\ \mathbf{K}_t \Big\}
\end{equation}

其中关键变量修正定义如下:
\begin{itemize}
    \item $\lambda_t = \frac{D_t}{FV_t}$:\textbf{统一杠杆率}。分母采用特许经营权价值(Franchise Value)$FV_t$,与后续终值函数 $FV_T-D_T$ 保持一致;
    \item $s_t$:\textbf{所有者持股比例}(Owner's Share),取值 $s_t\in[0,1]$,初始 $s_0=1$,随股权激励 $a_{\text{equity}}$ 逐年摊薄;
    \item $\mathbf{K}_t$:\textbf{经营配置状态}(Configuration State)。记录当前生效的票价、营销、分红、薪资和债务档位,用于保证 Action Masking 的马尔可夫性:
    \begin{equation}
        \mathbf{K}_t = [k_{\text{ticket}},\ k_{\text{marketing}},\ k_{\text{equity}},\ k_{\text{salary}},\ k_{\text{debt}}]
    \end{equation}
    \item $\mathbf{\Psi}_t$:薪资与合同结构向量(含 $\mu_{S,t},\ \sigma_{S,t},\ r_{\text{max},t},\ S_{\text{commit},t}$);
    \item $V_t$:本期品牌估值增长率(Valuation Growth Rate)。
\end{itemize}

\subsection{4. 环境状态 $\mathcal{E}_t$}
包含 8 个核心外生变量(含扩军标记 $I_{\text{Expansion}}$,与休赛期阶段 $\Theta_t=\text{Offseason}$ 结合触发扩军选秀逻辑):
\begin{equation}
\mathcal{E}_t = \Big\{ \text{Macro}_t,\ \text{Cap}_{\text{Growth}},\ I_{\text{Expansion}},\ T_{\text{MediaDeal}},\ \mu_{\text{size}},\ \text{Compete}_{\text{local}},\ N_{\text{Star,FA}},\ \text{Bidding}_t \Big\}
\end{equation}

\section{MDP 动作空间与决策流定义}

为了克服传统“概念列表式”动作空间维度可变、稀疏性高的问题,本文将决策变量标准化为**可计算的低维离散动作向量** $\mathbf{a}_t$。这使得问题可直接适配于 DQN 或 PPO 等强化学习算法。

\subsection{1. 统一动作向量定义}
在任意时间步 $t$,动作向量定义为 6 维整型向量:
\begin{equation}
\mathbf{a}_t = \{a_{\text{roster}},\; a_{\text{salary}},\; a_{\text{ticket}},\; a_{\text{marketing}},\; a_{\text{debt}},\; a_{\text{equity}}\}
\end{equation}
每个分量对应一个离散化的决策档位,具体定义如下:

\subsection{2. 分量详解}

\subsubsection{1) 竞技构建决策 $a_{\text{roster}}$ (Roster Strategy)}
决定球队在转会市场、选秀大会或交易截止日的操作方向与力度。
\begin{itemize}
    \item \textbf{空间定义}:$a_{\text{roster}} \in \{0, 1, 2, 3, 4\}$
    \begin{itemize}
        \item 0 (\textbf{Seller-Aggressive}): “清仓重建”。积极送出老将换取选秀权($\downarrow \mathbf{Q}_t, \uparrow \mathbf{C}_t$)。
        \item 1 (\textbf{Seller-Conservative}): “微调避税”。送出特定合同以降低薪资。
        \item 2 (\textbf{Hold}): “保持现状”。不进行主动交易操作。
        \item 3 (\textbf{Buyer-Conservative}): “补强”。用次轮签或中产特例补充角色球员。
        \item 4 (\textbf{Buyer-Aggressive}): “梭哈争冠”。送出未来资产换取即战力巨星($\uparrow \mathbf{Q}_t, \downarrow \mathbf{C}_t$)。
    \end{itemize}
    \item \textbf{阶段约束}:在 $\Theta_t = \text{Playoff}$ 或非交易期,该分量被强制锁定为 2 (Hold)。
\end{itemize}

\subsubsection{2) 薪资空间管理 $a_{\text{salary}}$ (Spending Target)}
设定本周期的目标薪资支出水平,间接控制签约动作。
\begin{itemize}
    \item \textbf{空间定义}:$a_{\text{salary}} \in \{0, 1, 2, 3\}$
    \begin{itemize}
        \item 0 (\textbf{Floor}): 仅满足穷人线(Cap Min),追求利润最大化。
        \item 1 (\textbf{Over-Cap}): 超过工资帽但低于税线,使用全额中产。
        \item 2 (\textbf{Taxpayer}): 进入奢侈税区间,保留核心阵容。
        \item 3 (\textbf{Apron}): 突破硬工资帽(Apron),不惜重金缴纳罚款打造豪门。
    \end{itemize}
\end{itemize}

\subsubsection{3) 票务与商业定价 $a_{\text{ticket}}$ (Ticket Pricing)}
\begin{itemize}
    \item \textbf{空间定义}:$a_{\text{ticket}} \in \{0, 1, 2, 3, 4\}$,对应倍率 $M_{\text{ticket}} \in \{0.9,\ 1.0,\ 1.1,\ 1.2,\ 1.3\}$
    \item \textbf{通过机制}:票价 = $\text{BasePrice} \times M_{\text{ticket}}$。
    \item \textbf{权衡}:高倍率提升短期 $Rev_{\text{gate}}$,但降低上座率并长期损害 $\mu_{\text{size}}$(球迷基础)。
\end{itemize}

\subsubsection{4) 市场营销投入 $a_{\text{marketing}}$ (Marketing Investment)}
\begin{itemize}
    \item \textbf{空间定义}:$a_{\text{marketing}} \in \{0, 1, 2\}$
    \begin{itemize}
        \item 0 (\textbf{Low}): 维持最低曝光,无额外支出。
        \item 1 (\textbf{Medium}): 投入营收的 5\% 用于推广,平稳提升 $V_t$。
        \item 2 (\textbf{High}): 投入营收的 10\% 进行全美推广(或流媒体合作),显著提升 $V_t$ 和 $\text{MediaExposure}$。
    \end{itemize}
\end{itemize}

\subsubsection{5) 债务杠杆调整 $a_{\text{debt}}$ (Leverage Control)}
\begin{itemize}
    \item \textbf{空间定义}:$a_{\text{debt}} \in \{0, 1, 2\}$
    \begin{itemize}
        \item 0 (\textbf{Deleverage}): 优先偿还债务($\Delta D < 0$)。
        \item 1 (\textbf{Maintain}): 借新还旧,维持当前杠杆率。
        \item 2 (\textbf{Leverage Up}): 激进融资($\Delta D > 0$),用于球场建设或支付巨额薪资。
    \end{itemize}
\end{itemize}

\subsubsection{6) 股权激励授予 $a_{\text{equity}}$ (Equity Dilution)}
针对题目核心特征(WNBA球员权益),决定稀释多少股权给予球员。
\begin{itemize}
    \item \textbf{空间定义}:$a_{\text{equity}} \in \{0, 1, 2, 3\}$,对应授予比例 $\omega \in \{0\%,\ 1\%,\ 2\%,\ 5\%\}$
    \item \textbf{作用}:授予越高,$\mathbf{\Psi}_t$ 中的现金薪资压力越小,球员满意度/留队率越高,但所有者剩余价值(Terminal Value)受损。
\end{itemize}

\subsection{3. 动作有效性与阶段约束 (Action Masking \& Markov Consistency)}
为确保决策的现实可行性且不破坏 Markov 性,我们使用\textbf{配置状态} $\mathbf{K}_t$ 作为约束基准,而非依赖上一时刻动作 $a_{t-1}$。
只有在当前阶段允许改变的动作分量才可变,其余分量必须强制等于当前配置状态 $\mathbf{K}_t$ 对应的值。

\begin{table}[h]
\centering
\caption{各赛季阶段允许的动作维度}
\begin{tabular}{|l|c|c|c|c|}
\hline
\textbf{动作分量} & \textbf{Offseason (休赛期)} & \textbf{Regular (常规赛)} & \textbf{TradeDeadline (交易期)} & \textbf{Playoff (季后赛)} \\ \hline
$a_{\text{roster}}$ & \checkmark & - & \checkmark & - \\ \hline
$a_{\text{salary}}$ & \checkmark & - & \checkmark & - \\ \hline
$a_{\text{ticket}}$ & \checkmark & - & - & - \\ \hline
$a_{\text{marketing}}$ & \checkmark & \checkmark & \checkmark & \checkmark \\ \hline
$a_{\text{debt}}$ & \checkmark & - & - & - \\ \hline
$a_{\text{equity}}$ & \checkmark & \checkmark & - & - \\ \hline
\end{tabular}
\end{table}

\textbf{掩码机制}:
\begin{equation}
\mathcal{A}_{\text{valid}}(s_t) = \Big\{ \mathbf{a} \in \mathcal{A} \ \Big|\ \forall j,\ \text{Mutable}(j, \Theta_t) \ \lor\ a_j = \mathbf{K}_t[j] \Big\}
\end{equation}
这确保了在冻结期(如季后赛),Agent 只能延续上一时刻的状态配置($a_j = k_j$),消除了对 $a_{t-1}$ 的非马尔可夫依赖。

\section{系统动力学:状态转移方程}

\subsection{1. 赛季阶段演化 (Phase Dynamics)}
阶段变量 $\Theta_t$ 遵循确定性的事件驱动逻辑:$\Theta_{t+1} = \text{Next}(\Theta_t)$。

\subsection{2. 竞技状态转移 (Competitive Dynamics)}
(此部分保持不变:$\mathbf{Q}_{t+1}$ 受交易影响,$\mathbf{A}_{t+1}$ 自然老化,$\mathbf{L}_{t+1}$ 合同滚动,战绩受 ELO 驱动)

\subsection{3. 财务状态转移 (Financial Dynamics)}
为彻底解决变量定义冲突,本节严格遵循财务状态空间(Section 11)的定义体系,引入**所有者权益分离**逻辑。

\subsubsection{(1) 现金流双轨制:运营 vs 储备}
\textbf{经营性现金流 (CFO)} $CF_t$ 是衡量当期盈利能力的核心指标,用于奖励函数:
\begin{equation}
\text{CF}_t = \text{Rev}_t(\mathbf{a}_t) - \text{Cost}_{\text{ops}} - \text{Interest}_t - \text{Tax}_t
\end{equation}
其中 $\text{Cost}_{\text{ops}}$ 包含运营支出与现金薪资:
\begin{equation}
\text{Cost}_{\text{ops}} = \text{Ops}_{\text{fixed}} + S_{\text{total}} \cdot (1 - \omega(a_{\text{equity}}))
\end{equation}
注意:此处 $\omega$ 体现了股权对现金薪资的替代作用(PMA条款),直接提升 $CF_t$(分子),从而提升当期奖励。

\textbf{现金储备 (Cash Reserves)} 的更新需包含融资活动,使用\textbf{净现金流}:
\begin{equation}
\text{Cash}_{t+1} = \text{Cash}_t + \text{CF}_t + \underbrace{\Delta D(a_{\text{debt}}) - \text{Repayment}}_{\text{融资活动}}
\end{equation}
\textbf{关键区分}:奖励函数只奖励 $CF_t$(赚来的钱),不奖励 $\Delta D$(借来的钱),彻底杜绝了“借债刷分”的漏洞。

\subsubsection{(2) 价值体系重构:企业价值 vs 所有者权益}
为避免重复计算稀释,我们将价值分解为三个独立变量:
\begin{itemize}
    \item \textbf{特许经营权价值 (Franchise Value, FV)}:即企业整体价值(Enterprise Value)。
    \begin{equation}
    FV_{t+1} = FV_t \times (1 + V_t)
    \end{equation}
    其中 $V_t$ 仍为增值率。
    
    \item \textbf{归属所有者份额 (Owner's Share, $s_t$)}:
    初始 $s_0=1$。每期股权授予 $a_{\text{equity}}$ 会导致永久性稀释:
    \begin{equation}
    s_{t+1} = s_t \times (1 - \omega(a_{\text{equity}}))
    \end{equation}
    
    \item \textbf{债务存量 (Debt, $D_t$)}:
    独立演化:$D_{t+1} = D_t + \Delta D$。
\end{itemize}

\subsubsection{(3) 杠杆率更新}
\begin{equation}
\lambda_{t+1} = \frac{D_{t+1}}{FV_{t+1}}
\end{equation}
此处与 Section 11 保持严格一致,分母为企业价值。

\subsubsection{(4) 配置状态更新}
配置状态 $\mathbf{K}_t$ 仅在动作可变的阶段被更新,否则保持不变(Freeze):
\begin{equation}
K_{j, t+1} = \begin{cases} 
a_{j, t} & \text{若 } \text{Mutable}(j, \Theta_t) \\
K_{j, t} & \text{否则 (Frozen)}
\end{cases}
\end{equation}
这一机制确保了即使动作 $a_t$ 在下一时刻改变,如果处于冻结期,系统状态也不会响应非法变化;结合 Action Masking,实际上 Agent 只能选择 $a_t = K_t$。

\subsection{4. 环境状态转移 (Environmental Dynamics)}
(保持马尔可夫链驱动逻辑不变)

---
% =================================================================================================
% SECTION 13: REWARD FUNCTION (REDESIGNED)
% =================================================================================================
\section{目标函数与奖励结构}

\subsection{1. 单步奖励函数 $R_t$}
\begin{equation}
R_t = w_{\text{CF}} \cdot \frac{\text{CF}_t}{\sigma_{\text{CF}}} 
+ w_{\text{val}} \cdot V_t 
+ w_{\text{win}} \cdot \mathbb{I}(\Theta_t=\text{Playoff})
- P_{\text{risk}}(\lambda_t)
\end{equation}
\textbf{严格定义}:
\begin{itemize}
    \item $CF_t$:Section 12.3.1 定义的经营性现金流(不含债权本金);
    \item $\sigma_{\text{CF}}$:\textbf{现金流归一化常数}(Scaling Factor)。取历史财务数据的标准差(如 \$5M),用于消除量级差异;
    \item $V_t$:\textbf{品牌估值增长率}(Valuation Growth Rate)。直接奖励资产增值速度;
    \item $P_{\text{risk}}(\lambda_t)$:统一使用 $\lambda_t = \frac{D_t}{FV_t}$ 计算的风险惩罚项。
\end{itemize}

\subsection{2. 终值函数 (Terminal Value)}
这是模型最关键的修正。为反映“所有者最终能拿走的财富”,我们计算**归属所有者的净权益**:
\begin{equation}
\text{TerminalValue}(S_T) = s_T \times \left( FV_T - D_T \right)
\end{equation}
其中 $s_T\in[0,1]$ 为所有者最终持股比例,$FV_T$ 与 $D_T$ 为终止时刻的特许经营权价值与债务存量,均属于财务状态 $\mathcal{F}_T$ 的分量。

\textbf{逻辑闭环}:
\begin{itemize}
    \item $FV_T - D_T$ 是最终时刻的**总权益价值 (Total Equity)**。
    \item $s_T$ 是经过 $T$ 年连续稀释后的**所有者持股比例**。
    \item 此公式清晰、简洁且完全避免了重复扣减的问题。Agent 必须权衡:是用股权(降低 $s_T$)来换取薪资空间(提升 $Win\% \to FV_T$),还是保留股权(维持 high $s_T$)但承担现金压力。
\end{itemize}

